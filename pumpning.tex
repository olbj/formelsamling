\chapter{Pumpsystem}
\section*{Beteckningar}

\acrfull{Vv}\\
\acrfull{Vv}\\
\acrfull{Vv}\\
\acrfull{Vv}\\
\acrfull{W} \\
\acrfull{Wprick} \\
\acrfull{A} \\
\acrfull{rho} \\
\acrfull{d} \\
\acrfull{xi} \\
\acrfull{hjd} \\
\acrfull{L} \\
\acrfull{m} \\
\acrfull{mprick} \\
\acrfull{NPSH} \\
\acrfull{hfl} \\
\acrfull{T} \\
\acrfull{p} \\
\acrfull{lambdat} \\
\acrfull{g} \\
\acrfull{n} \\
\acrfull{V} \\
\acrfull{v} \\
\acrfull{Vprick} \\
\acrfull{eta} \\
\acrfull{Hjd}

\section*{Formler och samband}
	\textbf{Teoretiskt pumparbete vid en isokor process: } $w=v \cdot \Delta p$ \par
\textbf{Energiekvationen }  
	\begin{align*}
    & p_1 + \varrho \cdot g \cdot h_1 + \dfrac{v_1^2}{2} \cdot \rho +  \Delta p_{pump} = \\ 
    & = p_2 + \varrho \cdot g \cdot h_2 + \dfrac{v_2^2}{2} \cdot \rho +  \Delta p_{f12} 
	\end{align*}
\textbf{Uppfordringshöjd:} 
	\begin{align*}
    H_{pump} & = \dfrac{p_{ut}-p_{in}}{\varrho \cdot g}+\dfrac{v^2_{ut}-v^2_{in}}{2\cdot g}+h_{ut}-h_{in} + \Delta h_f = &\\
    & = \dfrac{p_{0,ut}-p_{0,in}}{\varrho \cdot g}+h_{ut}-h_{in} + \Delta h_f  
	\end{align*}
$p_{ut}-p_{in}$ = ökningen av det statiska trycket, $p_{0,ut}-p_{0,in}$ = ökningen av totaltrycket i pumpen. $\Delta h_f$ = friktionsförluster i rörsystemet. \par
\textbf{Kavitation och NPSH}
 \begin{align*}
	& \textit{NPSH} = \dfrac{p_{0,s} - p_\text{å}}{\varrho \cdot g} = \dfrac{p_a - p_\text{å}}{\varrho \cdot g} - h_s - h_{fs} \\
	&	\textit{NPSH} \geq \textit{NPSH}_{erf}
 \end{align*}
$p_{0,s}$ = absoluta totaltrycket vid pumpinloppet, $p_\text{å}$ = vätskans ångbildningstryck vid aktuell temperatur, $p_a$ = lufttryck vid en öppen vätskeyta, $h_s$ = höjdskillnad mellan vätskeyta och pumpinlopp, samt $h_{fs}$ = strömningsförluster i sugledningen. $\textit{NPSH}_{erf}$ är pumpens minsta NPSH vid ett givet driftfall.\par
	\textbf{Pumparbete: }
	\begin{align*}
    & \dot{W} = \dfrac{\dot{W}_n}{\eta_{tot}} = \dfrac{\dot{V}\cdot \varrho \cdot g \cdot{H}}{\eta_{tot}}= \dfrac{\dot{m}\cdot g \cdot H }{\eta_{tot}}=  
    \dfrac{\dot{V}\cdot\Delta p_t}{\eta_{tot}} & 
	\end{align*}
       där $ \Delta p_t $  är totala tryckökningen i pumpen. \par
	\textbf{Likformighetslagar } 
	\begin{align*} & \dot{V} \propto n \cdot D^3 & \\ 
	& \dot{m} \propto n \cdot \varrho \cdot  D^3 \\ 
  & H\propto n^2 \cdot D^2  \\
  & \dot{W}\propto n^3 \cdot \varrho \cdot D^5  
	\end{align*}
	\textbf{Strömningsförluster: } 
	\begin{equation*}
		h_f \approx konst \cdot \dot{V}^2
	\end{equation*}
	\textit{Strömingsförluster som beror på friktion i cirkulära rör:}
	\begin{flalign*}
	&	h_f = \dfrac{\lambda \cdot L \cdot v^2_m}{2 \cdot d \cdot g} =  \dfrac{8 \cdot \lambda \cdot L \cdot \dot{V^2}}{\pi^2 \cdot d^5 \cdot g} \\
	&	p_f = \rho \cdot g \cdot h_f = \rho \cdot \dfrac{\lambda \cdot L \cdot v^2_m}{2 \cdot d } =  \rho \cdot \dfrac{8 \cdot \lambda \cdot L \cdot \dot{V^2}}{\pi^2 \cdot d^5 }
	\end{flalign*}
		där $v_m$ är fluidens medelhastighet \par 
	\textit{Strömningsförluster som beror på engångsförluster i ventiler, rörböjar etc:} 
	\begin{flalign*}
	&	h_f = \dfrac{\xi \cdot v^2_m}{2 \cdot g}\\	
	&	p_f = \rho \cdot g \cdot h_f = \rho \cdot \dfrac{\xi \cdot v^2_m}{2}	
	\end{flalign*}
 där $\xi$ är summan av engångsförlustkoefficienterna för systemet. 
