\chapter{Ekonomi}
\section*{Beteckningar}
\acrfull{når}\\
\acrfull{Au}\\
\acrfull{fA}\\
\acrfull{GI}\\
\acrfull{qi}\\
\acrfull{LCC}\\
\acrfull{ai}\\
\acrfull{Ci}\\
\acrfull{fN}\\
\acrfull{N}\\
\acrfull{Rvd}\\
\acrfull{r}\\
\acrfull{SV}\\
\acrfull{TauP}

\section*{Index}
	\begin{tabularx}{\linewidth} { l
	>{\raggedright\arraybackslash\hsize=1.5\hsize\linewidth=\hsize}X
	>{\raggedright\arraybackslash\hsize=0.5\hsize\linewidth=\hsize}X}
	\textit{0} & Värde vid tidpunkt 0 (idag)\\ 
  \textit{n} & Värde vid tidpunkt n\\ 
	\textit{N} & Nuvärde 
	\end{tabularx}

\section*{Formler och samband}

\textbf{Nuvärde}
\begin{equation*}
  N=K_n \cdot (1 + r)^{-n} = SN \cdot (1 + r)^{-n}
\end{equation*}

\textbf{Nuvärdesmetoden}
\begin{align*}
  & N = \sum_{i=1}^{n} C_i \cdot (1+r)^{-i} \\
  & N = \sum_{i=1}^{n} a \cdot (1+r)^{-i} = f_N \cdot a \\
  & \text{om det årliga intäktsöverskottet är konstant.} \\
  & f_N = \dfrac{1-(1+r)^{-n}}{r} = \dfrac{1}{f_A}
\end{align*}

\textbf{Annuitetsmetoden}
\begin{align*}
  & A = f_A \cdot N \\
  & f_A = \dfrac{r}{1-(1+r)^{-n}} = \dfrac{1}{f_N}
  % & N = \sum_{i=1}^{n} C_i \cdot (1+r)^{-i} \\
  % & N = \sum_{i=1}^{n} a \cdot (1+r)^{-i} = f_N \cdot a \\
  % & \text{om det årliga intäktsöverskottet är konstant.} \\
\end{align*}

\textbf{Slutvärde}
\begin{equation*}
 K_n= SV =N \cdot (1 + r)^n
\end{equation*}

\textbf{Realränta}
\begin{equation*}
  r_r = \dfrac{1+r}{1+q} \approx r - q 
\end{equation*}
där $r$ är den nominella räntan eller kalkylräntan.

\textbf{Paybackmetoden}
\begin{align*}
  & \text{Generellt: }\sum_{i=0}^{T} C_i = 0 \\
  & \text{Diskonterad paybackmetod: }\sum_{i=0}^{T} C_i \cdot (1+r)^{-i} = 0 \\
  &  T = \dfrac{G}{a} \text{ om det årliga intäktsöverskottet är konstant. } 
\end{align*}

\textbf{Life Cycle Cost (LCC)}
\begin{align*}
  LCC = G + \sum_{i=0}^{n} K_{N,i} - \sum_{i=0}^{n} I_{N,i} - R_N
\end{align*}
