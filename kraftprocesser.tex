\chapter{Kraft- och kylprocesser}
\section*{Beteckningar}
	\begin{tabularx}{\linewidth} { l
	>{\raggedright\arraybackslash\hsize=1.5\hsize\linewidth=\hsize}X
	>{\raggedright\arraybackslash\hsize=0.5\hsize\linewidth=\hsize}X}
	\acrshort{COP} & \acrlong{COP}\\
	\acrshort{eta} & \acrlong{eta}\\
	\acrshort{h} & \acrlong{h}\\
	\acrshort{mprick} & \acrlong{mprick}\\
	\acrshort{np} & \acrlong{np}\\
	\acrshort{Qprick} & \acrlong{Qprick}\\
	\acrshort{kompressionsf} & \acrlong{kompressionsf}\\
	\acrshort{tryckf} & \acrlong{tryckf}\\
	\acrshort{s} & \acrlong{s}\\
	\acrshort{v} & \acrlong{v}\\
	\acrshort{w} & \acrlong{w}\\
	\acrshort{x} & \acrlong{x}
\end{tabularx}
\section*{Ångkraftprocessen:}
\subsection*{Index}
	\begin{tabularx}{\linewidth} { l
	>{\raggedright\arraybackslash\hsize=1.5\hsize\linewidth=\hsize}X
	>{\raggedright\arraybackslash\hsize=0.5\hsize\linewidth=\hsize}X}
	\textit{f} & fuktig ånga\\ 
	\textit{is} & isentrop\\ 
	\textit{t1} & tillstånd efter kondensor\\ 
	\textit{t2} & tillstånd efter matarvattenpump\\ 
	\textit{t3} & tillstånd efter överhettare\\ 
	\textit{t4} & tillstånd efter turbin\\ 
	\textit{'} & mättad vätska; ingående flöde\\ 
	\textit{''} & mättad ånga; utgående flöde\\ 
	\end{tabularx}
	\textbf{Ånghalt:}
	\begin{align*}
		x = \dfrac{h_f-h'}{h''-h'} = \dfrac{s_f-s'}{s''-s'} = \dfrac{v_f-v'}{v''-v'}
	\end{align*}
	\textbf{Entalpi i fuktiga området:}
	\begin{align*}
		h_f = x \cdot h'' + (1  - x)  \cdot h' = x \cdot (h'' - h') + h' = x \cdot r + h' 
	\end{align*}
	\textbf{Termisk verkniningsgrad:}
	\begin{align*}
		\eta_t=\dfrac{(h_{t3}-h_{t4})-(h_{t2}-h_{t1})}{h_{t3}-h_{t2}}=1-\dfrac{h_{t4}-h_{t1}}{h_{t3}-h{t2}}
	\end{align*}
	\textbf{Turbinens isentropverkningsgrad:}
	\begin{align*}
		\eta_{is}=\dfrac{h_{t3}-h_{t4}}{h_{t3}-h_{t4,is}}
	\end{align*}
	\textbf{Värmebalans förvärmare:}
	\begin{align*}
		\dot{m}_{1,in}\cdot h_{1,in}+\dot{m}_{2,in}\cdot h_{2,in} = \dot{m}_{1,ut}\cdot h_{1,ut}+\dot{m}_{2,ut}\cdot h_{2,ut}
	\end{align*}
	\textbf{Massbalans blandningsförvärmare:}
	\begin{align*}
		\dot{m}_{1,in}+\dot{m}_{2,in} = \dot{m}_{3,ut}
	\end{align*}
\section*{Förbränningsmotorer}
\subsection*{Ideal Ottomotor}
\begin{align*}
	&r = \dfrac{V_{max}}{V_{min}} \\
	&V_{max} = \text{Volymen då kolven är i sitt nedersta läge} \\
	&V_{min} = \text{Volymen då kolven är i sitt översta läge} \\
	&\eta = 1- \dfrac{1}{r^{\kappa - 1}} \text{ då } c_p = \textit{konstant}
\end{align*}
\subsection*{Ideal Dieselmotor}
\begin{align*}
	&r_c = \dfrac{V_{3}}{V_{2}} \\
	&V_{3} = \text{Volymen då den isobara värmetillförseln är avslutad} \\
	&V_{2} = \text{Volymen då den isobara värmetillförseln börjar}\\
	&\text{Kompressionsförhållandet är } r = \dfrac{V_{max}}{V_{min}} \text{ där} \\
	&V_{max} = \text{ volymen innan kompressionen och} \\
	&V_{min} = \text{ volymen efter kompressionen} \\
	&\eta_{Diesel} = 1 -  \dfrac{1}{r^{(\kappa - 1)}} \cdot \left[\dfrac{r_c^\kappa - 1}{\kappa \cdot (r_c -1)} \right] \text{ då } c_p = \textit{konstant}
\end{align*}
\subsection*{Ideala gasturbiner}
\subsubsection*{Enkel gasturbincykel}
\begin{align*}
&\eta_t = \dfrac{w}{q_{23}}  = \dfrac{c_p \cdot (T_3 - T_2) - c_p\cdot (T_4 - T_1)} {c_p \cdot (T_3 - T_2)} = 1 - \dfrac{T_4 - T_1}{T_3 - T_2} \\
&\text{där netto arbete } w = \text{ skillnad mellan tillfört och avgivet värme}\\
&\left( \dfrac{T_2}{T_1} \right) = \left( \dfrac{p_2}{p_1} \right)^{\frac{\kappa -1 }{\kappa}} = {r_p}^{\frac{\kappa -1 }{\kappa}} \text{där } r_p = \dfrac{p_2}{p_1}\\
& \text{Ekvationen ovan kan omforumleras till}\\
&\eta_t = 1 - \dfrac{T_1}{T_2} = 1 - \dfrac{T_4}{T_3} = 1 - \dfrac{1}{\left( \dfrac{p_2}{p_1} \right)^{\frac{\kappa - 1}{\kappa}}}  =  1 - \dfrac{1}{r_p^{\frac{\kappa - 1}{\kappa}}} 
\end{align*}
\subsubsection*{Rekuperativ ideal gasturbin}
\subsubsection*{Index}
	\textit{1}: tillstånd före kompression\\ 
	\textit{2}: tillstånd efter kompressor\\ 
	\textit{3}: tillstånd före turbin\\ 
	\textit{4}: tillstånd efter turbin\\ 
	\textit{5}: tillstånd före brännkammare\\ 
	\textit{6}: tillstånd före rekuperator
	\begin{align*}
		\eta_t & = \dfrac{w}{q_{53}} = \dfrac{q_{\text{tillfört}} - q_{\text{bortfört}}}{q_{\text{brännkammare}}} \\
		= &  \dfrac{c_p \cdot (T_3 - T_2) - c_p \cdot (T_4 - T_1)}{c_p \cdot (T_3 - T_5)} \\
		= &  \dfrac{(T_3 - T_2) - (T_4 - T_1)}{T_3 - T_5} = \bigg| T_5 = T_4 \bigg| = \dfrac{(T_3 - T_4) - (T_2 - T_1)}{T_3 - T_4} \\
		= & 1- \dfrac{T_2 - T_1}{T_3 - T_4} = 1 - \dfrac{T_2}{T_3} = 1 - \dfrac{T_2}{T_1} \cdot \dfrac{T_1}{T_3}\\ 
			& 1 - \dfrac{1}{\dfrac{T_3}{T_1}} \cdot \left(\dfrac{p_2}{p_1}\right)^\frac{\kappa -1}{\kappa} \\
			& \text{då } T_5 = T_4
	\end{align*}
\subsubsection*{Verkliga processer}
Vid beräkning av verkliga processer kan polytropexponenten användas för att beskriva verklig kompression och expansion. Det är inte självklart att alla steg har samma polytropexponent. Om olika steg har skilda polytropexponenter måste varje steg beräknas separat. 
\section*{Kylmaskiner och värmepumpar}
\subsubsection*{Index}
	\begin{tabularx}{\linewidth} { l
	>{\raggedright\arraybackslash\hsize=1.5\hsize\linewidth=\hsize}X
	>{\raggedright\arraybackslash\hsize=0.5\hsize\linewidth=\hsize}X}
	\textit{is} & isentrop\\ 
	\textit{t1} & tillstånd efter förångare\\ 
	\textit{t2} & tillstånd efter kompressor\\ 
	\textit{t3} & tillstånd efter kondensor\\ 
	\textit{t4} & tillstånd efter ventil\\ 
	\textit{L} & Tillfört värme i förångaren\\ 
	\textit{H} & Avgivet värme i kondensorn\\ 
	\textit{R} & Kylmaskin\\ 
	\textit{HP} & Värmepump\\ 
	\end{tabularx}
	\textbf{Värmebalanser:}
	\begin{align*}
		{\dot{Q}}_{L} &  ={\dot{m}}_{\textit{köldmedium}}  \cdot (h_{t1} - h_{h4}) \\
		{\dot{Q}}_{H} &  ={\dot{m}}_{\textit{köldmedium}}  \cdot (h_{t2} - h_{h3}) \\
	\end{align*}
	\textbf{Köldfaktor och värmefaktor:}
	\begin{align*}
		\textit{COP}_{R}  & =  \dfrac{{\dot{Q}}_{L}}{{\dot{W}}_{kompressor}}  \\
		\textit{COP}_{HP}  & =  \dfrac{{\dot{Q}}_{H}}{{\dot{W}}_{kompressor}}  \\
	\end{align*}
	\textbf{Kompressorns isentropverkningsgrad:}
	\begin{align*}
		\eta_is=\dfrac{h_{t2,is}-h_{t1}}{h_{t2}-h_{t1}}
	\end{align*}
