\chapter{Fysikaliska data för några viktiga ämnen}
% Här är en parantes () \\
\footnotesize
\begin{tabularx}{\textwidth}{ |l|X|X|X|X|X|X|X|   }
\hline
\multicolumn{8}{|c|}{\textbf{Fysikaliska data för några viktiga ämnen}} \\
\hline
Ämne &p &T &\textit{M}&$\varrho$ &c\textsubscript{p}& $\lambda$ & $\mu$\\
		 &\textit{bar}&\textit{°C} &$kg/kmol$&$(kg/m^3)$ &$kJ/(kg \cdot K)$& $W/(m \cdot K) $ & $Pa \cdot s \cdot 10^6$\\
	% Ämne&Tryck (bar)&Temperatur °C&c_p $ kJ/kg \cdot K$ & Densitet $kg/m^3$ &Värme-konduktivitet $W/(m \cdot K)$\\
% Ämne&Tryck (bar)&Temperatur °C&Specifik värmekapacitet $kJ/kg \cdot K$&Densitet $kg/m^3$&Värme-konduktivitet $W/(m \cdot K)$\\
\hline
Luft & 1 &20 & 28,96&1,189 &1,005 &0,0254&18,1\\
Vatten & 1 &20 & 18,016&988,2 &4,181 &0,597&1005\\
\hline
\end{tabularx}
