\chapter{Värmeöverföring}
\section*{Beteckningar}
\acrfull{A}\\
\acrfull{alfad}\\
\acrfull{d}\\
\acrfull{d}\\
\acrfull{frek}\\
\acrfull{Vv}\\
\acrfull{c}\\
\acrfull{c0}\\
\acrfull{cp}\\
\acrfull{ny}\\
\acrfull{mprick}\\
\acrfull{Nu}\\
\acrfull{Pr}\\
\acrfull{Re}\\
\acrfull{Esk}\\
\acrfull{stefboltz}\\
\acrfull{vågl}\\
\acrfull{Ov}\\
\acrfull{T}\\
\acrfull{delta}\\
\acrfull{Qprick}\\
\acrfull{U}\\
\acrfull{lambda}\\
\acrfull{Rv}\\
\acrfull{alfa}

\section*{Värmeledning}
\textbf{Värmeledning genom plan vägg: }
 \begin{align*}
	\dot{Q} & = - \lambda \cdot A \cdot \dfrac{dT}{dx} = \lambda \cdot A \dfrac{T_1-T_2}{\delta} = \\
		& = \dfrac{(T_1-T_2)}{\left(\dfrac{\delta}{\lambda \cdot A}\right)}  = \left| \dfrac{\delta}{\lambda \cdot A}=R \right| =\dfrac{T_1-T_2}{R}
 \end{align*}
 där $T_1$ och $T_2$ är temperaturena på väggens varma respektive kalla yta. \par
 \textbf{Värmeledning genom kompositvägg med konvektionsöverföring:} 
	\begin{align*}
		\dot{Q} &= U \cdot A \cdot (T_i - T_u) =\\
		& = \dfrac{(T_i-T_u) \cdot A}{\left(\dfrac{1}{\alpha_i}+\dfrac{\delta_1}{\lambda_1}+ \dfrac{\delta_2}{\lambda_2}+\cdots +\dfrac{\delta_n}{\lambda_n}+\dfrac{1}{\alpha_u}\right)} = \\
		 & = \dfrac{(T_i-T_u)}{\left(\dfrac{1}{\alpha_i \cdot A}+\dfrac{\delta_1}{\lambda_1 \cdot A}+ \dfrac{\delta_2}{\lambda_2 \cdot A}+\cdots +\dfrac{\delta_n}{\lambda_n\cdot A} +\dfrac{1}{\alpha_u \cdot A}\right)} = \\
		 & = \dfrac{(T_i-T_u)}{R_{tot}}\\
		U \cdot A &= \dfrac{1}{R_{tot}}
	\end{align*}
 där $T_i$ och $T_u$ är temperaturena i väggens varma respektive kalla omgivning, långt från väggen. \par
 \textbf{Termisk diffusivitet}
 \begin{align*}
&\alpha = \dfrac{\lambda}{\varrho \cdot c_p} \text{ m\textsuperscript{2}/s}
 \end{align*}
\section*{Konvektion}
\textbf{Dimensionslösa tal} 
	\begin{flalign*}
	& \textit{Reynolds tal: } Re  = \dfrac{v_m \cdot d_h}{\nu} = &   \\ 
	& = \left| \mu = \dfrac{\nu}{\varrho}	\right| =\dfrac{v_m \cdot d_h \cdot \varrho}{\mu}   \\
	& d_h \text{ (hydraliska diametern )} = \dfrac{4 \cdot \textit{tvärsnittsarean}}{\textit{våt omkrets}} = \dfrac{4 \cdot A}{O} \\
	& v_m \text{ (medelhastigheten )} = \dfrac{\dot{m}}{\varrho \cdot A} \\
	& \textit{Prandtls tal: } Pr = \dfrac{c_p \cdot \mu}{\lambda} = \dfrac{c_p \cdot \nu \cdot \varrho}{\lambda} = \dfrac{\nu}{\alpha} \\
	& \text{där } \alpha \text{ i uttrycket }  \frac{\nu}{\alpha} \text{ är ämnets termiska diffusivitet.}\\
	& \textit{Nusselts tal: } Nu = \dfrac{\alpha \cdot d_h}{\lambda}
	\end{flalign*}
	Korrelationer för \acrfull{Nu} finns till exempel i  \cite{soleimani_mohseni_formelsamling_2018} och \cite{alvarez_energiteknik_2006}
  \subsection*{Samband mellan tryckfall och värmeöverföringskoefficient för cirkuära rör}
\subsubsection*{Konstanta röregenskaper, variabelt volymflöde}
\begin{tabularx}{\linewidth} { l l l}
	% >{\raggedright\arraybackslash\hsize=1.9\hsize\linewidth=\hsize}X
	% >{\raggedright\arraybackslash\hsize=0.1\hsize\linewidth=\hsize}X}
	\textit{Laminärt flöde} & ~~~~~  & \textit{Turbulent flöde} \\ 
  $P_f \text{ } \propto \text{ } v \text{ } \propto \text{ } \dot{V}$ &&  $p_f \text{ } \propto \text{ } v^{1,75 \text{ till } 2}$\\
  $\alpha \text{ } \propto \text{ } v^{0 \text{ till } 0,3}$ &&  $\alpha \text{ } \propto \text{ } v^{0,8}$\\
	\end{tabularx}
\subsubsection*{Konstant volymflöde, variabel rördiameter}
\begin{tabularx}{\linewidth} {l l l} 
	\textit{Laminärt flöde} & ~~~~~  & \textit{Turbulent flöde} \\ 
	$p_f \text{ } \propto \text{ } d^{-4}$ && $p_f \text{ } \propto \text{ } d^{-(4,75 \text{ till } 5)}$\\
	$\alpha \text{ } \propto \text{ } d^{-1}$ && $\alpha \text{ } \propto \text{ } d^{-1,8}$\\
	\end{tabularx}
\section*{Strålning}
\subsubsection*{Våglängd}
\begin{align*}
	&\lambda = \dfrac{c}{f} \\
	& c = \dfrac{c_{0}}{n} \text{ där n = index}
\end{align*}
\subsubsection*{Svartkroppsstrålning}
\begin{align*}
	E_b(T)=\sigma \cdot T^4
\end{align*}
