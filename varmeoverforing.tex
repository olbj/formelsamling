\chapter{Värmeöverföring}
\section*{Beteckningar}
\acrfull{A}\\
\acrfull{alfad}\\
\acrfull{d}\\
\acrfull{d}\\
\acrfull{frek}\\
\acrfull{Vv}\\
\acrfull{c}\\
\acrfull{c0}\\
\acrfull{cp}\\
\acrfull{ny}\\
\acrfull{mprick}\\
\acrfull{Nu}\\
\acrfull{Pr}\\
\acrfull{Re}\\
\acrfull{Esk}\\
\acrfull{stefboltz}\\
\acrfull{vågl}\\
\acrfull{Ov}\\
\acrfull{T}\\
\acrfull{delta}\\
\acrfull{Qprick}\\
\acrfull{U}\\
\acrfull{lambda}\\
\acrfull{Rv}\\
\acrfull{alfa}

\section*{Värmeledning}
\textbf{Värmeledning genom plan vägg: }
 \begin{align*}
	\dot{Q} & = - \lambda \cdot A \cdot \dfrac{dT}{dx} = \lambda \cdot A \dfrac{T_1-T_2}{\delta} = \\
		& = \dfrac{(T_1-T_2)}{\left(\dfrac{\delta}{\lambda \cdot A}\right)}  = \left| \dfrac{\delta}{\lambda \cdot A}=R \right| =\dfrac{T_1-T_2}{R}
 \end{align*}
 där $T_1$ och $T_2$ är temperaturena på väggens varma respektive kalla yta. \par
 \textbf{Värmeledning genom kompositvägg med konvektionsöverföring:} 
	\begin{align*}
		\dot{Q} &= U \cdot A \cdot (T_i - T_u) =\\
		& = \dfrac{(T_i-T_u) \cdot A}{\left(\dfrac{1}{\alpha_i}+\dfrac{\delta_1}{\lambda_1}+ \dfrac{\delta_2}{\lambda_2}+\cdots +\dfrac{\delta_n}{\lambda_n}+\dfrac{1}{\alpha_u}\right)} = \\
		 & = \dfrac{(T_i-T_u)}{\left(\dfrac{1}{\alpha_i \cdot A}+\dfrac{\delta_1}{\lambda_1 \cdot A}+ \dfrac{\delta_2}{\lambda_2 \cdot A}+\cdots +\dfrac{\delta_n}{\lambda_n\cdot A} +\dfrac{1}{\alpha_u \cdot A}\right)} = \\
		 & = \dfrac{(T_i-T_u)}{R_{tot}}\\
		U \cdot A &= \dfrac{1}{R_{tot}}
	\end{align*}
 där $T_i$ och $T_u$ är temperaturena i väggens varma respektive kalla omgivning, långt från väggen. \par
 \textbf{Termisk diffusivitet}
 \begin{align*}
&\alpha = \dfrac{\lambda}{\varrho \cdot c_p} \text{ m\textsuperscript{2}/s}
 \end{align*}
\section*{Konvektion}
\textbf{Dimensionslösa tal} 
	\begin{flalign*}
	& \textit{Reynolds tal: } Re  = \dfrac{v_m \cdot d_h}{\nu} = &   \\ 
	& = \left| \mu = \dfrac{\nu}{\varrho}	\right| =\dfrac{v_m \cdot d_h \cdot \varrho}{\mu}   \\
	& d_h \text{ (hydraliska diametern )} = \dfrac{4 \cdot \textit{tvärsnittsarean}}{\textit{våt omkrets}} = \dfrac{4 \cdot A}{O} \\
	& v_m \text{ (medelhastigheten )} = \dfrac{\dot{m}}{\varrho \cdot A} \\
	& \textit{Prandtls tal: } Pr = \dfrac{c_p \cdot \mu}{\lambda} = \dfrac{c_p \cdot \nu \cdot \varrho}{\lambda} = \dfrac{\nu}{\alpha} \\
	& \text{där } \alpha \text{ i uttrycket }  \frac{\nu}{\alpha} \text{ är ämnets termiska diffusivitet.}\\
	& \textit{Nusselts tal: } Nu = \dfrac{\alpha \cdot d_h}{\lambda}
	\end{flalign*}
	Korrelationer för \acrfull{Nu} finns till exempel i  \cite{soleimani_mohseni_formelsamling_2018} och \cite{alvarez_energiteknik_2006}
  \subsection*{Samband mellan tryckfall och värmeöverföringskoefficient för cirkuära rör}
\subsubsection*{Konstanta röregenskaper, variabelt volymflöde}
\begin{tabularx}{\linewidth} { l l l}
	% >{\raggedright\arraybackslash\hsize=1.9\hsize\linewidth=\hsize}X
	% >{\raggedright\arraybackslash\hsize=0.1\hsize\linewidth=\hsize}X}
	\textit{Laminärt flöde} & ~~~~~  & \textit{Turbulent flöde} \\ 
  $P_f \text{ } \propto \text{ } v \text{ } \propto \text{ } \dot{V}$ &&  $p_f \text{ } \propto \text{ } v^{1,75 \text{ till } 2}$\\
  $\alpha \text{ } \propto \text{ } v^{0 \text{ till } 0,3}$ &&  $\alpha \text{ } \propto \text{ } v^{0,8}$\\
	\end{tabularx}
\subsubsection*{Konstant volymflöde, variabel rördiameter}
\begin{tabularx}{\linewidth} {l l l} 
	\textit{Laminärt flöde} & ~~~~~  & \textit{Turbulent flöde} \\ 
	$p_f \text{ } \propto \text{ } d^{-4}$ && $p_f \text{ } \propto \text{ } d^{-(4,75 \text{ till } 5)}$\\
	$\alpha \text{ } \propto \text{ } d^{-1}$ && $\alpha \text{ } \propto \text{ } d^{-1,8}$\\
	\end{tabularx}
\section*{Strålning}
\subsubsection*{Våglängd}
\begin{align*}
	&\lambda = \dfrac{c}{f} \\
	& c = \dfrac{c_{0}}{n} \text{ där n = index}
\end{align*}
\subsubsection*{Fotonenergi}
\begin{align*}
	& e = h \cdot v = \dfrac{h \cdot c}{\lambda}\\
	& \text{där} \ h = \text{plancks konstant} = 6,626069 \cdot 10^{-34} J \cdot s
\end{align*}
\subsubsection*{Svartkroppsstrålning}
\begin{align*}
	E_b(T)=\sigma \cdot T^4 \
	\text{där } \sigma = \text{Stefan Boltzmanns konstant} = 5,67 \cdot 10^{-8} \ W/(m^2 \cdot K^4) 
\end{align*}
\subsubsection*{Emissivitet}
\begin{align*}
	\varepsilon(T)	= \dfrac{E(T)}{E_b(T)} = \dfrac{\int_{0}^\infty \varepsilon_\lambda(\lambda, T) \cdot E_{b,\lambda}(T,\lambda) \cdot d\lambda}{\sigma \cdot T^4}
\end{align*}
\subsubsection*{Absorptans, reflektans och transmittans}
% 25-02-28. Symbolerna nedan är inte med i symbollistan
\begin{align*}
	& \text{Absorptans} = \alpha = \dfrac{\textit{absorberad strålning}}{\textit{total strålning}} \\
	& \text{Reflektans} = \rho = \dfrac{\textit{reflekterad strålning}}{\textit{total strålning}} \\
	& \text{Transmittans} = \tau = \dfrac{\textit{transmitterad strålning}}{\textit{total strålning}} \\
	& \alpha = \dfrac{G_{abs}}{G} \\
	& \rho = \dfrac{G_{ref}}{G} \\
	& \tau = \dfrac{G_{trans}}{G} \\
	&	\text{där} \ G = \text{totalt inkommande strålning} \ (W/m^2) \\
	& G_{abs} + G_{ref} + G_{trans} = G \\
	& \alpha + \rho + \tau = 1 \\
	& \text{Opak kropp} \ \tau = 0 \rightarrow \alpha + \rho = 0 \\
	& \text{Svart kropp:} \ \tau=0; \rho = 0; \alpha = \varepsilon = 1 \\
	& \text{Kirchoffs lag:} \ \varepsilon(T) = \alpha(T) \\ &\text{då en yta är i termisk jämvikt och omsluten av en svart kropp}
\end{align*}
\subsubsection*{Vinkelfaktorer}
\begin{align*}
	& F_{i \rightarrow j} = \ \text{andelstrålning som lämnar yta i och träffar yta j} \\
	& \sum\limits_{j=1}^N F_{ij} = 1 \ \text{där ytan i är helt omsluten av N ytor.} \\ 
	& F_{11} \text{är strålning som yta i utbyter med sig själv}\\
	\\
	& \text{Superpositionsregeln:} \\
	& F_{1 \rightarrow (2+3)} = F_{12} + F_{13} \\
	& A_1  \cdot F_{1 \rightarrow (2+3)} = A_1 \cdot F_{12} + A_1 \cdot F_{13} \\
	& (A_1 + A_2) \cdot F_{1 \rightarrow (2+3)} = A_2 \cdot F_{12} + A_3 \cdot F_{13} \\
	\\
	& \text{Reciprocitetslagen} \\
	& A_1 \cdot F_{12} = A_2 \cdot F_{21}
\end{align*}
\subsubsection*{Värmeutbyte mellan svarta ytor:}
\begin{align*}
	& \dot{Q}_i = \sum\limits_{j=1}^N A_i \cdot F_{ij} \cdot \sigma \cdot \left( T_i^4 - T_j^4 \right) \\
	%& \text{Nettovärme från yta 1 till yta 2:} \\
	%& \dot{Q}_{1 \rightarrow 2} = A_1 \cdot \sigma \cdot T_1^4 \cdot F_{12} - A_2 \cdot \sigma \cdot T_2^4 \cdot F_{21} = \\
	%& =  \left| A_1 \cdot F_{12} = A_2 \cdot F_{21} \Rightarrow F_{21} = \dfrac{A_1 \cdot F_{12}}{A_2} \right| = A_1 \cdot \sigma \cdot  T_2^4 \cdot F_{12} 
\end{align*}
