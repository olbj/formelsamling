\chapter{Underhållsteknik}
\section*{Beteckningar}
\acrfull{Autg}\\
\acrfull{lambdaFI}\\
\acrfull{Kutg}\\
\acrfull{MRT}\\
\acrfull{MWT}\\
\acrfull{MDT}\\
\acrfull{MTTF}\\
\acrfull{MTBF}\\
\acrfull{MTTR}\\
\acrfull{Tlg}\\
\acrfull{TAK}

	\section*{Definitioner}
	Begrepp och nyckeltal definieras i svensk standard SS-EN 13306 och SS-EN 15341. Dessa är sammanfattade i SSG rapport "Underhållseffektivitet, terminilogi och nyckeltal".
	
	\subsection*{Definitioner enlig SS\_EN standard}
	
	Viktiga definioner enlig SSG rapport "Underhållseffektivitet, terminologi och nyckeltal (2015) är:
\begin{itemize}	
  \item Enhet: Del, komponent, anordning, delsystem, funktionell apparat, utrustning eller system som kan individuellt beskrivas och beaktas. 
  \item Fel: Upphörande av förmågan hos en enhet att utföra en krävd funktion. 
  \item Funktionsfel: Tillstånd hos en enhet karakteriserat av oförmåga att utföra en krävd funktion, exkluderat en oförmåga som kan uppstå vid förebyggande underhåll eller annan planerad åtgärd eller brist på stödfunktioner
	\item Underhåll: Kombination av alla tekniska, administrativa och ledningens åtgärder under en enhets livstid avsedda att vidmakthålla den i, eller återställa den till, ett sådant tillstånd att den kan utföra krävd funktion.
  % \item Förebyggande underhåll: Underhåll som genomförs vid förutbestämda intervall eller enligt förutbestämda kriterier och i avsikt att minska sannolikheten för fel eller degradering av en enhets funktion.
  \item Förebyggande underhåll: Underhåll som genomförs vid förutbestämda intervall eller enligt förut\-bestämda kriterier och i avsikt att minska sannolikheten för fel eller degradering av en enhets funktion.
	\item Förutbestämt underhåll: Förebyggande underhåll som genomförs i enlighet med bestämda intervaller eller efter en bestämd användning, men utan att föregås av tillståndskontroll.
	\item Tillståndsbaserat underhåll: Förebyggande underhåll som består av kontroll och övervakning av en enhets tillstånd avseende dess funktion och egenskaper, samt därav föranledda åtgärder
	\item Förutsägbart underhåll: Tillståndsbaserad underhållsåtgärd som genomförs som följd av en förutsägelse om en enhets försämrade funktion baserad på analys och utvärdering av viktiga egenskaper.
	\item Förbättring: Kombination av alla tekniska, administrativa samt ledningens åtgärder, avsedda att förbättra en enhets tillförlitlighet, utan att ändra dess krävda funktion.
	\item Modifiering: Kombination av alla tekniska, administrativa och ledningens åtgärder, avsedda att ändra en enhets funktion.
	\item Avhjälpande underhåll:Underhåll som genomförs efter det att funktionsfel upptäckts och med avsikt att få enheten i ett sådant tillstånd att den kan utföra krävd funktion.
	\item Uppskjutet avhjälpande underhåll: Avhjälpande underhåll som inte genomförs omedelbart efter att ett funktionsfel upptäckts utan senareläggs i enlighet med givna underhållsdirektiv.
	\item Akut avhjälpande underhåll: Underhåll som genomförs omedelbart efter det att funktionsfel upptäckts för att undvika oacceptabla konsekvenser
	\item Operatörsunderhåll: Underhåll som genomförs av en enhets användare eller operatör
\end{itemize}	
	
	
	\subsection*{Definitioner från Hagberg och Henriksson "Underhåll i världsklass"}
	
\begin{itemize}	
	\item Stopptid: = Väntetid + reparationstid
	\item Totalt disponibel tid\textsubscript{x}: = Obelagd tid + Totalt disponibel tid\textsubscript{y}
	\item Obelagd tid = planeringsfaktorn
	\item Totalt disponibel tid\textsubscript{y} = Planerad produktionstid
	\item Tillgänglig tid av planerad produktionstid är Totalt disponibel tid\textsubscript{y} - Stopptid
	\item MRT = Mean repair time: Tid att utföra reparation
	\item MWT = Mean waiting time: Väntetid fram till reparation
	\item MDT = Mean down time: Genomsnittligt stopptid
\end{itemize}	
	

	\section*{Nyckeltal}
	Standard SS-EN 15341 beskriver nyckeltalen i tre grupper:
	\begin{itemize}
	 \item E - Ekonomiska nyckeltal
	 \item O - Organisatoriska nyckeltal
	 \item T - Tekniska nyckeltal
	\end{itemize}
Nyckeltalen är sedan indelade i olika nivåer.
\subsection*{Ekonomiska nyckeltal}
		$\textbf{E1: } \dfrac{\text{Total underhållskostnad}}{\text{Återanskaffningsvärde}} \cdot 100$
			
		$\textbf{E3: } \dfrac{\text{Total underhållskostnad}}{\text{Total produktioh}}$

		$\textbf{E15: } \dfrac{\text{Kostnad för avhjälpande underhåll}}{\text{Total underhållskostnad}} \cdot 100$

		$\textbf{E16: } \dfrac{\text{Kostnad för förebyggande underhåll}}{\text{Total underhållskostnad}} \cdot 100$

		$\textbf{E17: } \dfrac{\text{Kostnad för tillståndsbaserat underhåll}}{\text{Total underhållskostnad}} \cdot 100$

		$\textbf{E18: } \dfrac{\text{Kostnad för förutbestämt underhåll}}{\text{Total underhållskostnad}} \cdot 100$

		$\textbf{E19: } \dfrac{\text{Kostnad för förbättring}}{\text{Total underhållskostnad}} \cdot 100$
\subsection*{Tekniska nyckeltal}
\subsubsection*{Enligt SS\_EN standard}
		$\textbf{T1: } \dfrac{\text{Total drifttid}}{\text{Total drifttid + Driftstopp på grund av underhåll}} \cdot 100$
	
		$\textbf{T17: } MTBF= \dfrac{\text{Total drifttid}}{\text{Totalt antal fel}} = MTTF + MDT$
	
		$\textbf{T21: } MTTR = MDT = \dfrac{\text{Total tid för återställande}}{\text{Totalt antal fel}} = MRT + MWT$
		\subsubsection*{Enligt Hagberg och Henriksson, Underhåll i världsklass}
		
			Operativ prestation: 
      \begin{align*}
        =\left( \dfrac{\text{Enheter}}{\text{Maskintimme}} \right) \cdot \left( \dfrac{\text{Maskintimmar}}{\text{År}} \right)= \text{ enheter/år}
      \end{align*}
			
			Tillgänglighet: 
      \begin{align*}
      T = \dfrac{\text{Totalt disponibel tid} - \text{Obelagd tid}}{\text{Totalt disponibel tid}}
      \end{align*}
			
			Tillgänglighet i seriekopplade funktioner: 
      \begin{align*}
      T = T_1 \cdot T_2 \cdot T_3 \cdot \ldots  \cdot T_n 
      \end{align*}
			
			Tillgänglighet i parallellkopplade funktioner: 
      \begin{align*}
      T = 1 - (1-T_1) \cdot (1-T_2) \cdot (1-T_3) \cdot \ldots  \cdot 1-T_n) 
      \end{align*}
			
			Funktionssäkerhet: 
      \begin{align*}
      MTTF = \dfrac{\text{Verkligt utnyttjad drifttid}}{\text{Antalet fel}} 
      \end{align*}
			
			Underhållsmässighet: 
      \begin{align*}
      MRT = \dfrac{\text{Total reparationstid}}{\text{Antalet fel}} 
      \end{align*}
			
			Underhållssäkerhet: 
      \begin{align*}
      MWT = \dfrac{\text{Total väntetid}}{\text{Antalet fel}} 
      \end{align*}
			
			Tillgängligheten: 
      \begin{align*}
      T = \dfrac{MTTF}{MTTF+MRT+MWT}
      \end{align*}

			Genomsnittligt stopptid: 
      \begin{align*}
      MDT = MTTR = MRT + MWT
      \end{align*}

			Felintensitet: 
      \begin{align*}
      \lambda = \dfrac{1}{MTTF} 
      \end{align*}

			textbf{Tillgängligheten: }
      \begin{align*}
      \lambda = \dfrac{1}{1+MDT\cdot\lambda} 
      \end{align*}
			
      Genomsnittsproduktion: 
      \begin{align*}
			 = \dfrac{\text{Totalt tillverkad volym under mätperioden}}{\text{Tillgänglig tid under mätperioden}} 
      \end{align*}
			
			\subsubsection*{T.A.K - beräkning}
			
      Tillgänglighet: 
      \begin{align*}
			T = \dfrac{\text{Totalt tillgänligt tid} - \text{Stopptid}}{\text{Totalt tillgänlig tid}}
      \end{align*}
			
      Anläggningsutnyttjande:  
      \begin{align*}
			A = \dfrac{\text{Bruttoproduktion}}{\text{Totalt tillgänlig tid}\cdot T \cdot \text{Maximal produktionshastighet}}
      \end{align*}
				
      Kvalitetsutbyte:  
      \begin{align*}
			K = \dfrac{\text{Bruttoproduktion} -\text{Defekt produktion}}{\text{Bruttoproduktion}}
      \end{align*}
				
		 Utrustningseffektivitet:  
      \begin{align*}
			TAK = T \cdot A \cdot K
      \end{align*}
			
\subsection*{Organisatoriska nyckeltal}
		\subsubsection*{Enligt SS\_EN standard}
		$\textbf{O4: } \dfrac{\text{Arbetstimmar underhåll som utförs av produktionsoperatör}}{\text{Totalt antal arbetstimmar som utförs av direkt underhållspersonal}} \cdot 100$

		$\textbf{O5: } \dfrac{\text{Planerade och schemalagda arbetstimmar för underhåll}}{\text{Toalt antal tillgängliga timmar}} \cdot 100$
		
		$\textbf{O9: } \dfrac{\text{Arbetstimmar underhåll som utförs av produktionsoperatör}}{\text{Totalt antal arbetstimmar som utförs av produktionsoperatörer}} \cdot 100$
		
		$\textbf{O16: } \dfrac{\text{Arbetstimmar avhjälpande underhåll}}{\text{Totalt antal arbetstimmar för underhåll}} \cdot 100$
		
		$\textbf{O17: } \dfrac{\text{Arbetstimmar akut avhjälpande underhåll}}{\text{Totalt antal arbetstimmar för underhåll}} \cdot 100$
		
		$\textbf{O18: } \dfrac{\text{Arbetstimmar förebyggande underhåll}}{\text{Totalt antal arbetstimmar för underhåll}} \cdot 100$
		
		$\textbf{O19: } \dfrac{\text{Arbetstimmar tillståndsbaserat underhåll}}{\text{Totalt antal arbetstimmar för underhåll}} \cdot 100$
		
		$\textbf{O20: } \dfrac{\text{Arbetstimmar förutbestämt underhåll}}{\text{Totalt antal arbetstimmar för underhåll}} \cdot 100$
\section*{Tillgänglighet}
			Tillgängligheten för en komponent: \\ 
      $ 0 \leqslant T \leqslant 1 $
			
			Systemtillgänglighet för seriekopplade komponenter:  \\ 
      $T_{system} = T_1 \cdot T_2 \cdot T_3 \cdots T_n$ 
			
			Systemtillgänglighet för parallellkopplade komponenter:  \\ 
      $ T_{system} = 1- \left(1 - T_1 \right) \cdot  \left(1 - T_2 \right) \cdot  \left(1 - T_3 \right) \cdots  \left(1 - T_n \right)$ 
