\documentclass[a4paper,11pt,onecolumn,fleqn]{report}
% \usepackage[a4paper, total={160mm, 250mm}]{geometry}
\usepackage[a4paper, width=150mm, top=25mm, bottom=25mm]{geometry}
% \usepackage[toc]{glossaries}
\usepackage[T1]{fontenc}
\usepackage[utf8]{inputenc}
\usepackage[swedish]{babel}
\usepackage{csquotes}
\usepackage[backend=biber, style=apa]{biblatex}
\DefineBibliographyStrings{swedish}{bibliography = {Referenser}}
\addbibresource{referenser.bib}
% \usepackage{hyphenat}
% \hyphenation{Mathe-matik wieder-gewinnen forut-bestamnda}
\usepackage{parskip}
\usepackage{setspace}
\usepackage[utf8]{inputenc}
\usepackage{import}
\usepackage[fleqn]{amsmath}
\usepackage{mathabx}
\usepackage{ltablex}
\usepackage{tabularx}
\usepackage{graphicx}
\usepackage[export]{adjustbox}
\usepackage[toc,acronym,nonumberlist]{glossaries}
\usepackage[yyyymmdd]{datetime}
\usepackage{hyperref}
\loadglsentries{symboler.tex}
\makeglossaries
%\newglossaryentry{testgl}{name={testgl},description={test glossary}}
%\newacronym{area}{A}{area m\textsuperscript{2}}
\author{Olof Björkqvist och Marcus Eriksson}
\title{Formelsamling i termodynamik och energiteknik}
\begin{document}
\import{./}{title.tex}
% \maketitle
\tableofcontents
\setlength\LTleft{0 cm}
\import{./}{general.tex}
\import{./}{varmelara.tex}
\import{./}{varmeoverforing.tex}
\import{./}{stromningslara.tex}
\import{./}{pumpning.tex}
\import{./}{varmevaxlare.tex}
\import{./}{kraftprocesser.tex}
\import{./}{ekonomi.tex}
\import{./}{underhallsteknik.tex}
% \import{./}{data.tex}
\cleardoublepage
\printglossary[title=Symboler, toctitle=Symboler, type=\acronymtype]
\cleardoublepage
\addcontentsline{toc}{chapter}{Referenser}
\printbibliography
\end{document}
